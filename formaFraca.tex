\subsection{Formulação fraca do Problema}

A forma fraca, ou forma variacional de uma equação diferencial é uma integral ponderada, equivalente à equação diferencial e às suas condições naturais de contorno. Por meio da forma fraca obtem-se um conjunto de equações algébricas relacionadas aos coeficientes desconhecidos ($c_i$). Para diferentes escolhas da função de peso, diferentes equações algébricas serão obtidas.
\citep[p. 64]{reddy}

A obtenção da forma fraca será ilustrada com a equação diferencial a seguir e suas condições de contorno:

\begin{equation}
	\label{eq:exFFraca}
	\mathcal{L}u = \frac{d^2 u}{dx^2} = f, \ u(a) = \alpha, \ u(b) = \beta
\end{equation}

Antes de se aplicar a discretização (ou interpolação) dada em \ref{eq:pppol}, substitui-se o problema contínuo a ser resolvido, na integral dos resíduos ponderados:

\begin{equation}
	R = \frac{d^2 u}{dx^2} - f = 0
\end{equation}

 \begin{equation}
	\int_{a}^{b} w\left(\frac{d^2 u}{dx^2} - f\right) dx = 0
 \end{equation} 
 
 Integrando por partes:
 
 \begin{equation}
 \label{eq:fFracaG}
 \left.w \frac{du}{dx}\right|_{a}^{b} - 
	\int_{a}^{b} 
	\frac{du}{dx} \frac{dw}{dx} dx = \int_{a}^{b} wf dx
 \end{equation} 

A equação \ref{eq:fFracaG} é denominada forma fraca da equação \ref{eq:exFFraca}. Adicionalmente, verifica-se a condição de contorno. Se for considerado que $ w(a) = w(b) = 0$ por exemplo, tem-se:

 \begin{equation}
 - 	\int_{a}^{b} 
	\frac{du}{dx} \frac{dw}{dx} dx = \int_{a}^{b} wf dx
 \end{equation} 
 
 Substituindo a função contínua $u$ por sua aproximação $U_N$ e considerando o método de Galerkin:
 
  \begin{equation}
  - \int_{a}^{b} 
 	\frac{d}{dx} 
 	\sum_{j = 1}^{N} c_j \phi_j
 	\frac{d}{dx} \phi_i dx = \int_{a}^{b} f\phi_i dx, \ i = (1, 2, ..., N)
  \end{equation} 
  
    \begin{equation}
    -\sum_{j = 1}^{N} \int_{a}^{b} 
   	\frac{d c_j \phi_j}{dx}    	 
   	\frac{d \phi_i}{dx}  dx = \int_{a}^{b} f\phi_i dx, \ i = (1, 2, ..., N)
    \end{equation} 
    
    \begin{equation}
    \label{eq:matFraca}
    \sum_{j = 1}^{N} 
    	A_{ij} c_j = F_i,  \ i = (1, 2, ..., N)
    \end{equation} 
    
A equação  ~\ref{eq:matFraca} é a forma matricial obtida com a aplicação do método de Galerkin, a qual é esparsa e simétrica.
