\subsection{Método Variacional}
Os métodos variacionais são técnicas utilizadas para extremizar o valor de um funcional em um determinado espaço de funções. Por extremizar entende-se encontrar o valor mínimo, máximo ou o ponto de inflexão do funcional.
Sistemas físicos, segundo o princípio de Hamilton ou princípio da mínima ação, tendem a minimizar sua energia interna ao realizar trabalho. Dessa forma, problemas envolvendo modelos matemáticos podem ser resolvidos considerando-se o problema de minimização de energia equivalente. (\textbf{Nota:} Um apêndice sobre elasticidade plana, princípio de Hamilton e análise funcional deve ser acrescentado).

Um funcional $ I(y) $ é uma regra que associa cada função $ u $ de um domínio $ \Omega $ a um único número real:

\begin{equation}
I : y \rightarrow \Re
\end{equation}

Em linhas gerais, pode ser entendido como uma função de funções. Um exemplo típico de funcional é a integral definida a seguir, a qual mapeia a função $ y $ em um valor real.

\begin{equation}
\label{eq:funcional}
I = \int_{a}^{b} u(y, y', x) dx
\end{equation}

Se uma função $ y(x) $ , por exemplo, minimiza o funcional, qualquer variação infinitesimal $ \alpha $ em $ y(x) $ produzirá um valor maior no funcional $ I $, o qual deixará de satisfazer a condição de mínimo. O operador variacional $ \delta $ desloca a função $ y $ em uma distância igual a $ \alpha \eta(x) $, sendo $ \alpha $ uma constante que tende a zero e $ \eta $ uma função suave definida em $[a,b]$, tal que $\eta(a) = \eta(b) = 0$.

Assim, tem-se: 

\begin{equation}
	\label{eq:sisEta}
	\begin{cases}
        y(x, \alpha) = y(x) + \alpha \eta(x) \\
        y(x, \alpha)' = y(x)' + \alpha \eta'(x) \\
		\alpha \eta(x) = \delta y(x) \\   
		\alpha \eta'(x) = \delta y'(x) \\  
    \end{cases}
\end{equation}

Se substituirmos as equações de \ref{eq:sisEta} em \ref{eq:funcional} o valor do funcional passa depender tanto da função $ y $ quanto de suas variações, realizadas pelo parâmetro $ \alpha $.

\begin{equation}
\label{eq:funcionalVar}
I(\alpha) = \int_{a}^{b} u(y(x, \alpha), y'(x, \alpha), x) dx
\end{equation}

A variação do funcional $ I $, pode ser entendida em termos diferenciais, como a derivada do funcional em relação ao incremento $\alpha$ da variação.

\begin{equation}
\delta I = \frac{\partial I}{\partial \alpha}
\end{equation}

O valor extremo procurado (mínimo, máximo ou ponto de inflexão) é caracterizado por possuir a primeira derivada igual a zero, portanto:

\begin{equation}
\delta I = \frac{\partial}{\partial \alpha} \int_{a}^{b} u(y(x, \alpha), y'(x, \alpha), x) dx = 0
\end{equation}

\begin{equation}
\delta I = \int_{a}^{b} \left(\frac{\partial u}{\partial y} \frac{\partial y}{\partial \alpha} + \frac{\partial u}{\partial y'} \frac{\partial y'}{\partial \alpha}\right) dx = 0
\end{equation}

Das equações em \ref{eq:sisEta} tem-se 

\begin{equation}
\begin{split}
\delta I = \int_{a}^{b} \left(\frac{\partial u}{\partial y} \eta + \frac{\partial u}{\partial y'} \frac{\partial }{\partial x}
\frac{\partial y}{\partial \alpha}\right) dx \\
= \int_{a}^{b} \left(\frac{\partial u}{\partial y} \eta + \frac{\partial u}{\partial y'} \eta'
\right) dx = 0
\end{split}
\end{equation}

É importante notar que a variação ocorre apenas no conjunto imagem do funcional, assim $\frac{\partial{x}}{\partial{\alpha}} = 0$. Integrando separadamente o segundo termo por partes tem-se:

\begin{equation}
\begin{split}
\delta I = \int_{a}^{b} \eta \left(\frac{\partial u}{\partial y} - \frac{d}{dx} 
\frac{\partial u}{\partial y'}\right)  dx +
\left[\frac{\partial u}{\partial y'} \eta(x) \right]_a^b
\end{split}
\end{equation}

Uma vez que a variação no contorno é zero (fronteiras fixas), $\eta(a) = \eta(b) = 0$, o valor de $ \delta I $ é dado por:

\begin{equation}
\label{eq:intEuler}
\begin{split}
\delta I = \int_{a}^{b} \eta \left(\frac{\partial u}{\partial y} - \frac{d}{dx} 
\frac{\partial u}{\partial y'}\right)  dx = 0
\end{split}
\end{equation}

Sendo $ \eta $ uma função derivável em $ [a,b] $, para que a igualdade \ref{eq:intEuler} seja satisfeita, torna-se necessário que o termo entre parentesis seja igual a zero

\begin{equation}
\label{eq:euler}
\frac{\partial u}{\partial y} - \frac{d}{dx} 
\frac{\partial u}{\partial y'} = 0, \ a < x < b
\end{equation}

Do conjunto de funções admissíveis, a única função que minimiza o funcional $I$  e portanto a energia interna do processo físico, é a mesma que resolve a equação \ref{eq:euler}, conhecida como Equação de Euler-Lagrange.
