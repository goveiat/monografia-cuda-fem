\subsection{Método dos Resíduos ponderados}

O método dos elementos finitos consiste em encontrar uma  solução do problema diferencial por meio da aproximação por polinômios. 
\citep[p. 97]{davis}
A aproximação de funções por polinômios é uma técnica conhecida, como por exemplo os métodos de interpolação de Newton e Lagrange e a aproximação por série de Taylor. 

Uma vez que o método opera sobre um domínio $ \Omega $ discretizado, de forma que $ \Omega \approx \sum \Omega_j $, no qual $ \Omega_e $ é um elemento discreto, os polinômios utilizados na aproximação devem ser definidos em partes, para cada elemento.

 O somatótrio definido na equação \ref{eq:pppol} aproxima a solução $ u $ de uma equação diferencial ou sistema de equações. As contantes $ c_j $ são variáveis desconhecidas, as quais devem ser encontradas de forma a resolver o problema. As funções $ \phi_j $ são conhecidas e diferenciaveis por partes no domínio $ \Omega $ e respeitam as condições de contorno.
 

\begin{equation}
	\label{eq:pppol}
	u(x) \approx U_N (x) = \sum_{j = 1}^{N} c_j \phi_j (x)
\end{equation}

 Seja uma equação denotada com o operador diferencial $ \mathcal{L} $, tal que
 
 \begin{equation}
 	\mathcal{L} u = f
 \end{equation}
 
 Uma vez que $U_N$ aproxima $u$ a relação a seguir é válida, e denotada por residual da aproximação
 
 \begin{equation}
 	R = \mathcal{L} u - f \neq 0
 \end{equation}

