\section{Método das diferenças finitas}

O método das diferenças finitas substitui cada derivada da equação diferencial por um quociente-diferença apropriado,
\citep[p. 684]{burden_faires}
isto é, obter aproximações discretas para cada derivada.

Sendo o domínio $ \Omega $ contínuo e limitado em $ x = {a, b} $, o mesmo pode ser discretizado escolhendo-se um valor $ N > 0 $ e dividindo-se o intervalo $ [a, b] $ em $ N + 1 $ subintervalos iguais, os quais são limitados por

\begin{equation}
  x_i = a + ih,  i = 0, 1, ... N+1
\end{equation}

O valor do incremento h é dado por 

\begin{equation}
  h = \frac{b-a}{N+1}
\end{equation}

Nos pontos interiores $ x_i , com i = 1, 2. .... N $, isto é, dentro da borda $ {a, b} $, a equação diferencial a ser aproximada é

\begin{equation}
    \label{eq:edo2}
    y''(x_i) = p(x_i)y'(x_i) + q(x_i)y(x_i) + r(x_i)
\end{equation}

Deseja-se obter o valor de $ y(x_i) $ como uma média do valor de seus vizinhos $ y(x_i + h) $ e $ y(x_i - h) $. Para tal, utiliza-se a expansão em série de Taylor, como por exemplo, a de ordem 3. A notação do grande O foi utilizada para denotar o erro do truncamento.

\begin{equation}
    \label{eq:dfr}
    y(x_i + h) = y(x_i) + hy'(x_i) + \frac{h^{2}}{2}y''(x_i) + \frac{h^{3}}{6}y'''(x_i) + O(h^{4})
\end{equation}

\begin{equation}
    \label{eq:dfl}
    y(x_i - h) = y(x_i) - hy'(x_i) + \frac{h^{2}}{2}y''(x_i) - \frac{h^{3}}{6}y'''(x_i) + O(h^{4})
\end{equation}

Somando-se as equações \ref{eq:dfr} e \ref{eq:dfl} e isolando a derivada procurada, no caso a de segunda ordem.

\begin{equation}
    \label{eq:difcent2}
    y''(x_i) = \frac{1}{h^2}[y(x_i + h) - 2y(x_i) +y(x_i - h)] - O(h^{4})
\end{equation}

A equação \ref{eq:difcent2} é denomidada fórmula da diferença centrada, pois obtém o valor de um dado ponto a partir do valor de todos os seus vizinhos. De forma similar, para a derivada de primeira ordem, obtém-se

\begin{equation}
    \label{eq:difcent}
    y'(x_i) = \frac{1}{2h}[y(x_i + h) - y(x_i - h)] - O(h^{2})
\end{equation}

Com as derivadas da equação \ref{eq:edo2} definidas, obtém-se então a sua forma discretizada. Os termos de ordem superior relativos ao erro de aproximação, foram desconsiderados na equação \ref{eq:edoDisc}.

\begin{equation}
    \label{eq:edoDisc}
   \frac{y(x_i + h) - 2y(x_i) +y(x_i - h)}{h^2} = p(x_i) \frac{y(x_i + h) - y(x_i - h)}{2h} +q(x_i)y(x_i) + r(x_i)
\end{equation}


O resultado obtido em  \ref{eq:edoDisc} pode ser expandido para M dimensões. As equações \ref{eq:lap} e \ref{eq:lapDisc} mostram a equação de laplace em duas dimensões e a mesma equação discretizada.

\begin{equation}
    \label{eq:lap}
    \Delta u = \frac{\partial^2 u}{\partial^2 x} + \frac{\partial^2 u}{\partial^2 y} = 0
\end{equation}

\begin{equation}
    \label{eq:lapDisc}
   \frac{u(x + h, y) - 2u(x, y) +u(x - h, y)}{h^2} +
   \frac{u(x, y + h) - 2u(x, y) +u(x, y - )}{h^2} = 0
\end{equation}