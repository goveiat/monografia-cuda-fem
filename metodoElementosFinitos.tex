\section{Método dos Elementos Finitos}

Segundo \citep[p. 19, 27]{jin}, o método dos elementos finitos (MEF) é uma técnica numérica que possibilita a aplicação das funções de interpolação, utilizadas na aproximação de PVC, sobre subdomínios do problema.

Já segundo \citep[p. 13]{reddy}, o MEF trata-se de um método no qual o domínio do problema é visto como uma coleção de subdomínios, chamados de elementos finitos, sobre os quais, a equação que modela o problema é aproximada por um método variacional ou de resíduos ponderados.



Como foi dito anteriormente, a resolução analítica de muitos problemas físicos se torna impraticável à medida em que a sua  complexidade aumenta. 
A fim de contornar esta dificuldade, estratégias numéricas para a discretização do domínio e aproximação da solução foram desenvolvidas por matemáticos, engenheiros e cientistas \citep[p. 1]{zien}. 

Dentre estas estratégias destacam-se o método das diferenças (MDF) finitas e o método dos elementos finitos (MEF), mostrados na figura \ref{fig:mdfFem}. O MDF consiste na discretização do domínio do problema por meio de uma grade de pontos e na aproximação de cada derivada da equação por um quociente-diferença adequado
\citep[p. 684]{burden_faires}. Embora este método seja útil em muitos casos, se torna difícil aplicá-lo em problemas com geometria irregular ou com condições de contorno não usuais. \citep[p. 4]{huebner}

\begin{figure}[!htb]
\centering
\includegraphics[scale=0.5]{figuras/temp.png}
\caption{Método das diferenças Finitas e método dos Elementos Finitos}
\label{fig:mdfFem}
\end{figure}

Diferentemente do MDF, como coloca \citep[p. 4]{huebner}, o MEF divide o domínio não em pontos, mas em subdomínios sobre os quais as equações são aproximadas por partes, e não pontualmente, como ocorre no MDF. O FEM também é capaz de representar mais fielmente o contorno (ou a borda) do problema. Desta forma, ele se apresenta como uma técnica mais poderosa e versátil para a modelagem de fenômenos com geometria complexa e meios não homogêneos \citep[p. 390]{sadiku}. 

O MEF surgiu originalmente como uma técnica de análise de deslocamentos e elasticidade de estruturas mecânicas, mas em seguida foi estendido para solucionar problemas de outros campos da física e da engenharia. \citep[p. 19]{jin} \citep[p. 3]{desai} \citep[p. 2]{zien}

As primeiras formulações do MEF são conhecidas como \textbf{abordagem direta} ou \textbf{formulação física}, que embora forneça a interpretação intuitiva do método, é util apenas para a resolução de problemas relativamente simples \citep[p. 6]{huebner}. O uso do princípio do trabalho virtual, para a determinação de forças na abordagem direta, levou à generalização do MEF, por meio da estratégia de minimização do funcional de energia. Esta técnica mais genérica  ficou conhecida como \textbf{formulação Variacional} \citep[p. 113]{desai} \citep[p. 20]{zien}. Uma terceira abordagem, conhecida como \textbf{Método dos Resíduos Ponderados} ou \textbf{MEF generalizado} \citep[p. 61]{zien} é tradicionalmente utilizada e é ainda mais genérica que o princípio Variacional, pois resolve diretamente as equações diferenciais do modelo, sem necessitar da existência de um funcional. \citep[p. 261]{desai}


\subsection{Sistema de Elementos Discretos}
\citep[p. 68]{desai} introduz o conceito de Método de Elementos Discretos (MED) como sendo uma etapa intermediária da formulação física MEF. Na representação em elementos discretos, cada elemento é unidimensional, de tal forma que o sistema completo é representado por uma estrutura aramada, conforme mostra a figura \ref{fig:arame}.

De forma similar, \citep[p. 2]{zien} apresenta o Sistema Discreto Padrão (SDP) como uma forma unificada  de analisar problemas de natureza discreta.


Estas abordagens são úteis na compreensão do funcionamento do MEF e do conceito de elemento.

\begin{figure}[!htb]
\centering
\includegraphics[scale=0.5]{figuras/temp.png}
\caption{Estruturas aramadas idealizadas}
\label{fig:arame}
\end{figure}



Chamado de 
 Sistemas discretos podem ser entendidos como a associação de partes funcionais para a formação de uma estrutura


\subsection{Etapas de Processamento}
A computação de um problema modelado a partir de elementos finitos compreende basicamente três etapas: 
\begin{enumerate}  
\item \textbf{Pré-processamento}: Entrada de Dados ou discretização;
\item \textbf{Processamento}: Análise do problema e  solução do sistema de equações;
\item \textbf{Pós-processamento}: Apresentação dos Resultados. 
\end{enumerate}

Cada etapa apresenta uma contribuição para que se obtenha um resultado satisfatório. O pré-processamento é responsável pela discretização e por estabelecar as restrições físicas do domínio. A etapa de análise obtém a aproximação do modelo (forma fraca), e aplica esta aproximação em cada sub-domínio. O resultado preliminar da análise é um sistema de equações lineares que quando resolvido, fornece a solução do problema.  O pós-processamento consiste na exibição adequada dos resultados.
\citep[p. 665, 666]{zien}


A seguir cada etapa é vista em detalhes e exemplificada com um sistema discreto padrão.

\subsubsection{Pré-Processamento}

A etapa de pré-processamento compreende a maior parte do tempo de modelagem do método. É neste ponto do processo que se são definidos os apectos geométricos do modelo, as propriedades dos materiais do objeto de estudo e a aplicação das condições de contorno ou condições iniciais.

\paragraph{Aspectos geométricos \\} 
A definição dos aspectos geométricos consiste na tranformação do domínio contínuo $\Omega$ em uma malha de elementos finitos. Cada elemento $\Omega_e$ dessa malha é um subdomínio de $\Omega,$. A figura \ref{fig:malhas} contém a representação de diferentes malhas propostas para o mesmo dominio $\omega$. A primeira e a segunda malha são uniformes, pois contém elementos de um único tipo, triângulos ou quadradados. Já a terceira malha é mista, pois apresenta elementos de ambos os tipos. 

\begin{figure}[!htb]
\centering
\includegraphics[scale=0.5]{figuras/temp.png}
\caption{Diferentes malhas para o domínio $\Omega$}
\label{fig:malhas}
\end{figure}

Cada elemento pode ser identificado na malha a partir de um número que lhe é atribuído. Na figura \ref{fig:numeracao}, os números entre parêntesis identificam os elementos.

Os vértices de uma elemento são chamados de nós. cada nó possui dois números vinculados a ele. Os índices na cor azul representam a numeração do nó dentro de um dado elemento (Identificação local). Os índices em vermelho por sua vez são os identificadores globais dos nós ao longo de toda a malha. Cada nó apresenta um determinado grau de liberdade dado pelo número máximo de componentes de um deslocamento. No exemplo de duas dimensões, cada nó tem no máximo 2 graus de liberdade, pois podem se deslocar nas direções de x e de y. O nó 1, no entanto, não apresenta graus de liberdade, pois está fixo. 

Os graus de liberdade de um elemento é a soma dos graus de liberdade de seus nós, dessa forma, um elemento triangular, por exemplo, no espaço bidimensional, possui no máximo 6 graus de liberdade.

\begin{figure}[!htb]
\centering
\includegraphics[scale=0.5]{figuras/temp.png}
\caption{Identificadores de Elementos e nós}
\label{fig:numeracao}
\end{figure}

\paragraph{Propriedades do material \\}
Para que o MEF seja capaz de aproximar adequadamente a solução de um problema, é necessário, além da geometria, se especificar as propriedades dos materiais em cada elemento da malha. Na análise estrutural, por exemplo, diferentes materiais reagem diferentemente à uma dada deformação ou  deslocamento, devido à sua rigidez e aos diferentes fenômenos que podem ocorrer, como por exemplo, fenômenos plásticos ou elásticos. Na análise eletromagnética, as propriedades dos materiais podem ser dadas em termos por exemplo, de condutividade, resistividade ou permissividade. Na análise térmica, algumas propriedades relevantes são o calor específico, a condutividade térmica e a expansão térmica.
Desta forma, de acordo com a área de análise, os materiais apresentam características determinantes para a obtenção de resultados adequados.

As propriedades dos materiais podem ainda estar condicionadas à uma determinada direção. Desta forma, os materiais podem ainda ser classificados como isotrópicos, anisotrópicos ou ortotrópicos. Materiais isotrópicos apresentam as mesmas propriedades físicas em todas as direções. Nos materiais anisotrópicos, as propriedades variam conforme a direção considerada. Nos materiais ortotrópicos, as propriedades são diferentes em direções perpendiculares entre si, formando eixos de ortotropia.

A título de exemplo, a tabela \ref{tab:permissividade} contém a comparação entre a permissividade relativa de diferentes materiais.

\begin{table}	
	\centering
	\begin{tabular}{|c|c|}	

	\end{tabular}
	\caption{A permissividade elétrica dos materiais}
	\label{tab:permissividade}
\end{table}


\paragraph{Valores de Contorno \\}
A atribuição dos valores de contorno é o último passo da etapa de pré- processamento. Estes valores são atribuídos a pontos específicos da malha e são importantes para caracterizar a unicidade de solução.

Considere um capacitor de placas paralelas, no qual uma das armaduras é submetida à uma tensão de 10V e outra à 0V. A ditribuição de potencial e o valor do campo elétrico entre as placas e no ambiente em volta do capacitor, podem ser definidos unicamente a partir das condições de contorno (10V e 10V) impostas em pontos (ou nós) específicos do domínio, como mostra a figura \ref{fig:capacitor}.

\begin{figure}[!htb]
\centering
\includegraphics[scale=0.5]{figuras/temp.png}
\caption{Distribuição do potencial e o campo elétrico no capacitor de placas paralelas}
\label{fig:capacitor}
\end{figure}

\subsubsection{Processamento ou Análise de Elementos Finitos}
A etapa de processamento de elementos finitos consiste na elaboração de funções aproximadas para o modelo em questão. Essas aproximações devem ser feitas dentro de cada sub-domínio do sistema. A forma forte do problema é geralmente fornecida como uma equação ou sistemas de equações diferenciais e deve ser transformada no processamento em um conjunto de equações algébricas contínuas por partes.

\paragraph{Funções sobre o espaço discreto \\}

Nesta seção será apresentada a visão geral do método. Para tal, os conceitos de formulação forte e fraca não serão necessários, visto que os problemas aqui descritos são modelados por sistemas lineares e não por equações diferenciais. Assim sendo, o MEF pode ser aplicado diretamente ao problema, sem a necessidade de se obter a forma integral (forma fraca) a partir de técnicas à relaxação da forma forte, como por exemplo o princípio dos trabalhos virtuais (abordagem direta), o método de Ritz (abordagem Variacional) e de Galerkin (abordagem por Resíduos ponderados).

Considere a malha bidimensional  apresentada na figura \ref{fig:malhaGenerica}. Esta malha pode representar por exemplo, a abstração de uma ponte metálica, de dutos de um fluido ou até mesmo um circuito elétrico, como apresentado na figura \ref{fig:repMalhaGenerica}.
\begin{figure}[!htb]
\centering
\includegraphics[scale=0.5]{figuras/temp.png}
\caption{Malha Genérica}
\label{fig:malhaGenerica}
\end{figure}

\begin{figure}[!htb]
\centering
\includegraphics[scale=0.5]{figuras/temp.png}
\caption{Possíveis objetos de representação da malha \ref{fig:malhaGenerica}}
\label{fig:repMalhaGenerica}
\end{figure}


 Fazendo uma analogia com uma estrutura metálica, conforme apresentado na figura \ref{fig:repMalhaGenerica}-a, as forças que atuam nos nós, podem ser definidas  a a partir dos deslocamento destes nós, da distribuição da carga sobre um dado elemento e da tensão inicial, caso haja.
 
 Supondo que um carregamtno $p$ atue sobre o elemento $1$, as forças que atuam sobre os nós deste elemento podem ser dadas por:
 
 
 \begin{equation}
 	\label{eq:forca}
 	\begin{tabular}{c c}
 	$q_1 = 
		\left \{
 		\begin{tabular}{c}
	 		$q_1^1$ \\
	 		$q_2^1$ \\
	 		$q_3^1$
  		\end{tabular} 		
		\right \}$
		\
 	$q_1^1 = 
		\left \{
 		\begin{tabular}{c}
	 		$U_1$ \\
	 		$V_1$
  		\end{tabular} 		
		\right \}	$
		\end{tabular} 	
 \end{equation}


Similarmente, os deslocamentos podem ser dados para cada nó do elemento. Se as componentes da força na direção de x e y são dadas como U e V, as componentes da força podem ser dadas em função dessas componentes de deslocamento.

 \begin{equation}
 	\label{eq:desloc}
 	\begin{tabular}{c c}
 	$u_1 = 
		\left \{
 		\begin{tabular}{c}
	 		$u_1^1$ \\
	 		$u_2^1$ \\
	 		$u_3^1$
  		\end{tabular} 		
		\right \}$
		\
 	$u_1^1 = 
		\left \{
 		\begin{tabular}{c}
	 		$u_1(U_1)$ \\
	 		$v_1(V_1)$
  		\end{tabular} 		
		\right \}	$
		\end{tabular} 	
 \end{equation}


Assumindo um material isotrópico, cujo comportamento é linear elástico, a lei de Hooke fornece a equação \ref{eq:hooke}

 \begin{equation}
 	\label{eq:hooke}
	\textbf{$q^1 = K^1 u^1 + f^1$}
 \end{equation}
 
 O vetor $q$ representa as forças induzidas pelos deslocamentos $u$ dos nós. A matriz $K$ é a matriz de rigidez ou matriz de coeficientes do problema. $f$ é a força ou tensão inicial do elemento. Caso o estado inicial do elemento seja de equilíbrio, $f$ é igual a zero.
 
 Para que a representação seja adequada é necessário que sejam estabelecidas condições de compatibilidade de deslocamentos e equilíbrio.
 
 A compatibilidade de deslocamento é necessária, uma vez que quando um nó de um elemento é deslocado em uma direção, os elementos vizinhos que compartilham o mesmo nó também são afetados com este deslocamento. Esta condição é satisfeita aos se relacionar todas as forças do sistema.
 
 \paragraph{Montagem do Sistema Global}
 
 Para que o conjunto de funções de força ou tensão sobre os subdomínios sejam corretamente agregadas em um sistema global, é necessário se estabelecer as condições de compatibilidade de deslocamento e de equilíbrio.
 
 Como um nó de um elemento é compartilhado com os elementos vizinhos, a compatibilidade dos deslocamentos ocorre por meio da montagem de um sistema global. Este sistema possibilita que todos os deslocamentos estejam relacionados entre si.
 
  A figura \ref{fig:loc2glob} mostra a transformação dos sistemas locais para um sistema com referências globais. Como todos os nós são relacionados entre si, a matriz de rigidez é quadrada e também simétrica para o caso de materiais isotrópicos.
  
  \begin{figure}[!htb]
  \centering
  \includegraphics[scale=0.5]{figuras/temp.png}
  \caption{}
  \label{fig:loc2glob}
  \end{figure}
  
  O sistema em \ref{eq:assembly} mostra a relação entre todos os pontos do domínio para um dado elemento. é importante notar que apenas elementos adjacentes se afetam mutuamente.
  
 \begin{equation}
 	\label{eq:assembly}
 	\begin{tabular}{c c c}
 	$q^e = 
		\left \{
 		\begin{tabular}{c}
	 		$q^e_1$ \\
	 		$q^e_2$ \\
	 		\vdots \\
	 		$q^e_m$
  		\end{tabular} 		
		\right \}$
		\
 	$u^e = 
		\left \{
 		\begin{tabular}{c}
	 		$u_1$ \\
	 		$u_2$ \\
	 		\vdots \\
	 		$u_m$
  		\end{tabular} 		
		\right \}	$
		\
		$K^e =
		\begin{bmatrix}
		    K^e_{11} 	& K^e_{12}  & \dots 	& K^e_{1m} \\
		    K^e_{11} 	& \ddots  & \ 	& \vdots \\
		    \vdots 	& \vdots  	 & \ 	& \vdots \\
		    K^e_{m1} 	& \dots   & \dots 	& K^e_{mm} 
		\end{bmatrix}	 $		
	\end{tabular} 
 \end{equation}
 
 Com os deslocamento relacionados em todos os pontos elementos do sistema,
 a condição de equilíbrio é satisfeita quando o somatório das forças causadoras de tais deslocamentos nula, ou seja, a resultante em um dado ponto $a$ vale zero.
 
  \begin{equation}
  	\label{eq:somaForcas}
 	\sum_{e=1}^{m}{q_a^e = q_a^1 + q_a^2 + \dots = 0}
  \end{equation}
  
  Considerando que o corpo está inicialmente em equilíbrio, ou seja, $f = 0$, o sistema como um todo pode ser representado como 
  
    \begin{equation}
    	\label{eq:equilibrio}
    	q =
   		\sum_{b=1}^{n}\sum_{e=1}^{m}{K_{ab}^e u_b = 0}
    \end{equation}
    
    
\paragraph{Atribuição das condições de Contorno \\}
A atribuição dos valores de contorno da variável $u$ consiste na especificação dos deslocamentos ou deformações prefixadas no sistema. No exemplo \ref{fig:repMalhaGenerica}-a, como análise estrutural, as condições de contorno podem ser os deslocamentos nulos impostos nós fixados (soldados) ou condições iniciais de tensão ou torção. a equação \ref{eq:condIni} exemplifica a atribuição dessas condições.

\begin{equation}
   	\label{eq:condIni}
 	u_1 = u_6 = 
		\left \{
 		\begin{tabular}{c}
	 		$0$ \\
	 		$0$ \\
  		\end{tabular} 		
		\right \}	
\end{equation}

A inserção de valores de contorno no sistema promove a redução do número de equações de equilíbrio, por meio da eliminação das linhas cujo valor de $u$ já foi especificado.

\paragraph{Exemplo \\}

A fim de exemplificar a etapa de processamento,  considere a malha de resistores introduzidas  em \ref{fig:repMalhaGenerica}-b. Neste caso, as condições de contorno são as a tensões fornecidas por uma bateria.

\subsubsection{Pós-Processamento}
A etapa de pós processamento consiste na apresentação dos resultados obtidos no processamento e na determinação de variáveis secundárias a partir destes resultados.

A apresentação da solução pode ser feita graficamente por diferentes técnicas, como curvas de nível, mapas vetoriais, sobreposição de imagens e animações. Alguns exemplos são dados na figura \ref{fig:graf}

\begin{figure}[!htb]
\centering
\includegraphics[scale=0.5]{figuras/temp.png}
\caption{}
\label{fig:graf}
\end{figure}

Com base nos valores apresentados no pós processamento, novas decisões são tomadas, tanto para otimizar os resultados quanto para melhorar a performance.

\subsection{Abordagem Direta \\}

A abordagem direta, abordagem física ou formulação de deslocamentos foi a primeira tentativa de se modelar um problema físico em termos de elementos finitos.

Voltada para a análise de estruturas e problemas de elasticidade, esta técnica busca aproximar o comportamento de um problema contínuo, por meio de elementos finitos, que se comporte de maneira similar a elementos reais discretos.
\citep[p. 19]{zien}

A partir desta abordagem, a forma fraca do problema diferencial é obtida com o uso do princípio dos trabalhos virtuais.
\citep[p. 20]{zien}

 

