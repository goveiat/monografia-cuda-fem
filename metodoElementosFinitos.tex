\section{Método dos Elementos Finitos}

Muitos problemas de valor de contorno não podem ser resolvidos por métodos analíticos, uma vez que surgem dificuldades ao lidar com características específicas do modelo, tais como coeficientes variáveis, regiões irregulares, condições de contorno inadequadas, existência de interfaces ou devido à grande quantidade de detalhes.
\citep[p. 410]{powers}

O método dos elementos finitos, é um método numérico, tal como o método das diferenças finitas, no entanto, é mais genérico e adequado às aplicações do mundo real. Neste método, o domínio do problema é visto como uma coleção de subdomínios, sobre os quais, a equação que modela o problema é aproximada por um método variacional.
\citep[p. 13]{reddy}


O método consiste em encontrar uma aproximação da solução por meio de polinômios definidos por partes. Dessa forma, o somatótrio definido em \ref{eq:pppol} aproxima a solução $ u $ de uma equação diferencial ou sistema de equações.
\citep[p. 97]{davis}

\begin{equation}
	\label{eq:pppol}
	u(x) \approx U_N (x) = \sum_{j = 1}^{N} c_j \phi_j (x)
\end{equation}

As contantes $ c_j $ são variáveis desconhecidas, as quais devem ser encontradas de forma a resolver o problema. As funções $ \phi_j $ são conhecidas e diferenciaveis por partes. Tais funções são conhecidas como função base ou função peso. A solução aproximada $ U_N $ consiste portanto

A aproximação de funções por polinômios é uma técnica conhecida, como por exemplo os métodos de interpolação de Newton e Lagrange e a aproximação por série de Taylor. 

\subsection{Método Variacional}
Os métodos variacionais são técnicas utilizadas para extremizar o valor de um funcional em um determinado espaço de funções. Por extremizar entende-se encontrar o valor mínimo, máximo ou o ponto de inflexão do funcional.

Um funcional $ I(u) $ é uma regra que associa cada função $ u $ de um domínio $ \Omega $ a um único número real:

\begin{equation}
I : u \rightarrow \Re
\end{equation}

Em linhas gerais, pode ser entendido como uma função de funções. Um exemplo típico de funcional é a integral definida a seguir, a qual mapeia a função $ u(x) $ em um valor real.

\begin{equation}
\label{eq:funcional}
I = \int_{a}^{b} u(y, y', x) dx
\end{equation}

Se uma função $ y(x) $ , por exemplo, minimiza o funcional, qualquer variação infinitesimal $ \alpha $ em $ y(x) $ produzirá um valor maior no funcional $ I $, o qual não satisfaz a condição de mínimo.

\begin{equation}
\delta y(x) = \alpha \eta(x), \ \  \alpha \rightarrow 0
\end{equation}

O operador variacional $ \delta $ desloca a função $ u $ em uma distância igual a $ \alpha \eta(x) $. O parâmetro $ \alpha $ do incremento é uma constante pequena e a função $ \eta $ , da mesma forma que $ y $, é definida em $[a,b]$, sendo $\eta(a) = \eta(b) = 0$.

Assim, tem-se 

\begin{equation}
	\label{eq:sisEta}
	\begin{cases}
        y(x, \alpha) = y(x) + \alpha \eta(x) \\
        y(x, \alpha)' = y(x)' + \alpha \eta'(x) \\
    \end{cases}
\end{equation}

Se substituirmos as equações de \ref{eq:sisEta} em \ref{eq:funcional} o valor do funcional passa relacionar tanto a função $ y $ quanto as suas aproximações a partir dos parâmetros $ \alpha $ e $ \eta $.

\begin{equation}
\label{eq:funcionalVar}
I(\alpha) = \int_{a}^{b} u(y(x, \alpha), y'(x, \alpha), x) dx
\end{equation}

A variação do funcional $ I $, pode ser denotada a partir da diferença entre o funcional deslocado em $ \alpha $ e o funcional sem o deslocamento, considerando que $\alpha $ tende a zero.

\begin{equation}
\delta I = I(\alpha) - I(0) = \frac{\partial I}{\partial \alpha}
\end{equation}

O valor extremo procurado (mínimo, máximo ou ponto de inflexão) é caracterizado por possuir a primeira derivada igual a zero, portanto

\begin{equation}
\delta I = \frac{\partial}{\partial \alpha} \int_{a}^{b} u(y(x, \alpha), y'(x, \alpha), x) dx = 0
\end{equation}

\begin{equation}
\delta I = \int_{a}^{b} \left(\frac{\partial u}{\partial y} \frac{\partial y}{\partial \alpha} + \frac{\partial u}{\partial y'} \frac{\partial y'}{\partial \alpha}\right) dx = 0
\end{equation}

Das equações em \ref{eq:sisEta} tem-se 

\begin{equation}
\begin{split}
\delta I = \int_{a}^{b} \left(\frac{\partial u}{\partial y} \eta + \frac{\partial u}{\partial y'} \frac{\partial }{\partial x}
\frac{\partial y}{\partial \alpha}\right) dx \\
= \int_{a}^{b} \left(\frac{\partial u}{\partial y} \eta + \frac{\partial u}{\partial y'} \eta'
\right) dx = 0
\end{split}
\end{equation}

Integrando separadamente o segundo termo por partes tem-se

\begin{equation}
\begin{split}
\delta I = \int_{a}^{b} \eta \left(\frac{\partial u}{\partial y} - \frac{d}{dx} 
\frac{\partial u}{\partial y'}\right)  dx +
\left[\frac{\partial u}{\partial y'} \eta(x) \right]_a^b
\end{split}
\end{equation}

Uma vez que $\eta(a) = \eta(b) = 0$, o valor de $ \delta I $ é dado por

\begin{equation}
\label{eq:intEuler}
\begin{split}
\delta I = \int_{a}^{b} \eta \left(\frac{\partial u}{\partial y} - \frac{d}{dx} 
\frac{\partial u}{\partial y'}\right)  dx = 0
\end{split}
\end{equation}

Sendo $ \eta $ uma função derivável em $ [a,b] $, para que a igualdade \ref{eq:intEuler} seja satisfeita, torna-se necessário que o termo entre parentesis seja igual a zero

\begin{equation}
\label{eq:euler}
\frac{\partial u}{\partial y} - \frac{d}{dx} 
\frac{\partial u}{\partial y'} = 0, \ a < x < b
\end{equation}

Do conjunto de funções admissíveis, a única função que minimiza o funcional dado em \ref{eq:funcional}, e que portanto satisfaz às condições do modelo, é a mesma que resolve a equação \ref{eq:euler}, conhecida como Equação de Euler.
