\section{Método dos Elementos Finitos}

Muitos problemas de valor de contorno não podem ser resolvidos por métodos analíticos, uma vez que surgem dificuldades ao lidar com características específicas do modelo, tais como coeficientes variáveis, regiões irregulares, condições de contorno inadequadas, existência de interfaces ou devido à grande quantidade de detalhes.
\citep[p. 410]{powers}

O método dos elementos finitos, é um método numérico, tal como o método das diferenças finitas, no entanto, é mais genérico e adequado às aplicações do mundo real. Neste método, o domínio do problema é visto como uma coleção de subdomínios, sobre os quais, a equação que modela o problema é aproximada por um método variacional.
\citep[p. 13]{reddy}


