\section{Método dos Elementos Finitos}

Muitos problemas de valor de contorno não podem ser resolvidos por métodos analíticos, uma vez que surgem dificuldades ao lidar com características específicas do modelo, tais como coeficientes variáveis, regiões irregulares, condições de contorno inadequadas, existência de interfaces ou devido à grande quantidade de detalhes.
\citep[p. 410]{powers}

O método dos elementos finitos, é um método numérico, tal como o método das diferenças finitas, no entanto, é mais genérico e adequado às aplicações do mundo real. Neste método, o domínio do problema é visto como uma coleção de subdomínios, sobre os quais, a equação que modela o problema é aproximada por um método variacional.
\citep[p. 13]{reddy}


O método consiste em encontrar uma aproximação da solução por meio de polinômios definidos por partes. Dessa forma, o somatótrio definido em \ref{eq:pppol} aproxima a solução $ u $ de uma equação diferencial ou sistema de equações.
\citep[p. 97]{davis}

\begin{equation}
	\label{eq:pppol}
	u(x) \approx U_N (x) = \sum_{j = 1}^{N} c_j \phi_j (x)
\end{equation}

As contantes $ c_j $ são variáveis desconhecidas, as quais devem ser encontradas de forma a resolver o problema. As funções $ \phi_j $ são conhecidas e diferenciaveis por partes. Tais funções são conhecidas como função base ou função peso. A solução aproximada $ U_N $ consiste portanto

A aproximação de funções por polinômios é uma técnica conhecida, como por exemplo os métodos de interpolação de Newton e Lagrange e a aproximação por série de Taylor. 

