% !TeX spellcheck = <none>
\section{Problema de Valor de Contorno}


Um problema bem-posto, segundo as definições de Hadamard (1923), é aquele que 	apresenta a existência, unicidade e estabilidade de solução. Para que estaas condições sejam atendidas, é necessário que o modelo matemático descreva adequadamente o fenômeno analisado e que as condições de contorno, ou condições iniciais, sejam bem estabelecidas. \textbf{***REF}

Em problemas de valor inicial, \textbf{PVI}, contém as condições iniciais do problema, as quais são impostas sobre a variável dependente e suas derivadas em um único instante $t_0$. Já em problemas de valor de contorno \textbf{PVC}, as restrições são dadas em pontos distintos, como por exemplo sobre $x_i$ e $x_f$. \citep[p. 447]{boyce_diprima}. Os sistemas de equações \ref{eq:pvi} e \ref{eq:pvc} mostram respectivamente um problema de valor inicial e de contorno, ambos de segunda ordem.

\begin{equation}
	\label{eq:pvi}
	\begin{cases}
		y'' + p(t)y' + q(t)y = f(t) \\
		y(t_0) = y_0 \\
		y'(t_0) = y_0'
	\end{cases}
\end{equation}


\begin{equation}
	\label{eq:pvc}
	\begin{cases}
		y''(x) + p(x)y'(x) + q(x)y(x) = f(x) \\
		y(x_i) = \alpha \\
		y'(x_f) = \beta
	\end{cases}
\end{equation}

As condições estabelecidas sobre a variável dependente, são condições \textbf{essenciais} ou de \textbf{Dirichlet}. As que são estabelecidas sobre as derivadas da variável dependente são  condições \textbf{naturais} ou de \textbf{Neumann}.

Além das restrições de Dirichlet e Neumann, existem aquelas que são específicas do fenômeno modelado, como por exemplo, condições de radiação ou de impedência para problemas do eletromagnetismo. \citep[p. 20]{jin}. A tabela \ref{tab:cond} apresenta alguns exemplos de condições de contorno.


\begin{table}	
	\centering
	\begin{tabular}{|c|c|}	
		\hline
		\textbf{Condição} 
		& \textbf{Tipo} \\	
		\hline
		$y(x_k) = y_k $ 
		& Dirichlet \\
		\hline
		$y(x_k) = 0$
		& Dirichlet Homogênea\  \\
		\hline
		$y'(x_k) = y_k$
		& Neumann \\
		\hline
		$y'(x_k) = 0$
		& Neumann Homogênea\  \\
		\hline
	\end{tabular}
	\caption{Exemplos de condições de contorno}
	\label{tab:cond}
\end{table}


Problemas físicos reais normalmente apresentam  condições de contorno complexas, existência de interfaces e grande quatidades de detalhes. Essas características tornam resolução analítica impraticável, sendo necessário recorrer a métodos numéricos para se obter uma solução aproximada.
\citep[p. 397]{powers}.


