\section{Problema de Valor de Contorno}


A solução de um problema regido por um modelo matemático depende de informações ligadas ao contexto em que tal problema ocorre.
Estas informações podem dizer a respeito das condições iniciais ou das condições de contorno. 
Condições iniciais caracterizam um \textbf{Problema de Valor Inicial (PVI)} e geralmente são relativas ao instante de tempo inicial do processo.
As condições de contorno por sua vez, caracterizam um \textbf{Problema de Valor de Contorno (PVC)} e  normalmente são dadas em função do limite espacial do processo em questão
\citep[p. 447]{boyce_diprima}.
Desta forma, enquanto os PVI geralmente relatam processos transientes, isto é, variantes no tempo, os PVC lidam com problemas em estado estacionário.

O PVI de um modelo de ordem $N$ é composto da equação do processo e do valor inicial da variável dependente e de suas derivadas, até a ordem $N-1$, como mostra a equação a seguir:

\begin{equation}
	\begin{cases}
		y'' + p(t)y' + q(t)y = f(t) \\
		y(t_0) = y_0 \\
		y'(t_0) = y_0'
	\end{cases}
\end{equation}

De forma análoga, o PVC é definido a partir da equação que modela o processo e suas condições nos pontos de contorno ou borda.

\begin{equation}
	\begin{cases}
		y'' + p(x)y' + q(x)y = f(x) \\
		y(x_i) = \alpha \\
		y'(x_f) = \beta
	\end{cases}
\end{equation}

Portanto, se as condições adicionais forem dadas em um único instante (ou ponto), tem-se o PVI do processo. Caso sejam dadas em dois ou mais pontos distintos, tem-se o PVC.

As condições de contorno usualmente podem ser classificadas como condições de Dirichlet ou de Neumann. 
As \textbf{condições de Dirichlet} são também conhecidas como \textbf{essenciais} e são estabelecidas sobre a variável dependente. Já as \textbf{condições de Neumann} são condições \textbf{naturais} e são impostas sobre as derivadas da variável dependente. A seguir são dados alguns exemplos.

\begin{equation}
	\begin{tabular}{l l}
		$y(x_k) = y_k $ 
		& Cond. Dirichlet \\
		$y(x_k) = 0$
		& Cond. Dirichlet Homogênea\  \\
		$y'(x_k) = y_k$
		& Cond. Neumann \\
		$y'(x_k) = 0$
		& Cond. Neumann Homogênea\  \\
	\end{tabular}
\end{equation}

A resolução analítica de PVC, ou mesmo de PVI, se torna impraticável à medida em que a complexidade do modelo, ou do contexto em que ele ocorre, aumenta.
Tal complexidade pode ocorrer, por exemplo, com a existência de coeficientes variáveis, regiões irregulares ou com condições de contorno inadequadas, existência de interfaces ou devido à grande quantidade de detalhes do problema
\citep[p. 410]{powers}.

Métodos numéricos apropriados podem ser utilizados na obtenção de uma solução aproximada para tais problemas. Neste trabalho serão abordados o \textbf{Método das Diferenças Finitas (MDF)} e o \textbf{Método dos Elementos Finitos (MEF)}.

